\documentclass[12pt]{article}
\usepackage[utf8]{inputenc}

\usepackage[
	citecounter,
	labelnumber,
	backend=biber,
	bibencoding=utf8,
	sorting=none
]{biblatex}
\addbibresource{./bibliography.bib}

\title{Complex Systems Project Proposal}
\author{
	Wayne Stegner % Laura's Brother
	\and
	Zuguang Liu
	\and
	Siddharth Barve
}
\date{\vspace{-36pt}}

\begin{document}
	\maketitle

	\vspace{-10pt}
	\section*{Goals and Aims of the Project}
	\vspace{-6pt}
	\par The goal of this project is to emulate the naturally emergent behavior
	of a penguin colony using swarm-based coordination.
	Our first aim is to design a set of behavioral rules for the individual
	penguins which result in emergent behavior similar to that of a natural
	penguin colony and maximizes survival.
	Our second aim, if time permits, is to evolve the rule set for the penguin
	colony to attain a similar emergent behavior as described in our first aim.
	We would like to do a thorough comparison of the rules defined in the first
	aim and those derived in the second aim.

	\vspace{-10pt}
	\section*{Description of the Proposed Work}
	\vspace{-6pt}
	\par In this project, we introduce a user-defined colony of penguins into
	an environment to observe social thermal-regulation of the population.
	Common (or similar) rules for individual penguins and the environment (with
	spatial and temporal variations) are set up such that the collective
	behavior of the penguin colony resembles that in nature.
	The specific rules will be explored empirically or algorithmically to
	simulate the natural constraints.
	These constraints will include heat transfer between individual penguins in
	the population and between penguins and their ambient environment.
	Heat transfer will be determined by the neighborhood of penguins.
	An additional constraint will be a nominal temperature range that must be
	maintained by the individual penguins for survival; prolonged exposure of
	temperatures outside of this range will result in death of the individual.
	Our hypothesis is their huddling pattern is formed by the egoistic survival
	instinct from the penguin individuals, and that this behavior can be
	realized through evolutionary methods.

	\vspace{-10pt}
	\section*{Motivation for the Project}
	\vspace{-6pt}
	\par Social thermal-regulation is an interesting animal behavior in social
	mammals, birds, and reptiles \cite{campbell_social_2018}.
	Understanding it helps to engineer energy-efficient solutions for us or
	artificial robots under extreme environments.
	Additionally, being able to evolve the behavior of the individual agents to
	achieve a complex collective behavior allows for more adaptive swarm
	robotic systems.
	We like penguins.
	Honestly, it's just that.

	\vspace{-10pt}
	\section*{Metrics for Evaluating Success or Failure}
	\vspace{-6pt}
	\par The success of the first aim is determined by whether the predefined
	rules result in a collective behavior that is favorable to the survival of
	the penguin group.
	The success will be proportional to the percent of surviving population
	after a user-defined period of time, as well as evaluating metrics such as
	the average warmth of the colony.
	A more subjective measure of evaluating the success of the project is to
	analyze the degree of emulation of natural penguin colony behavior.
	We will document the behavior of the penguins by exporting a 2D frame for
	each discrete time step which will be culminated into a GIF.
	The success of the second aim will be determined by the ability to achieve
	the success of the first aim, with a specified degree of error, or better.
	Additionally, the success of the second aim will be determined by
	improvement in simplicity, fewer hyper-parameters, and computation time.

	\vspace{-10pt}
	\printbibliography
\end{document}
