\documentclass[handout,9pt]{beamer}

% Choose how your presentation looks.
%
% For more themes, color themes and font themes, see:
% http://deic.uab.es/~iblanes/beamer_gallery/index_by_theme.html
%
\mode<presentation>
{
	\usetheme{Darmstadt}      % or try Darmstadt, Madrid, Warsaw, ...
	\usecolortheme{beaver} % or try albatross, beaver, crane, ...
	\usefonttheme{serif}  % or try serif, structurebold, ...
	\setbeamertemplate{navigation symbols}{}
	\setbeamertemplate{caption}[numbered]
}

% *** Formatting Packaages ***
\usepackage{bookmark}
\usepackage{hyperref}
\hypersetup{
	colorlinks=true,
	linkcolor=blue,
	filecolor=magenta,
	urlcolor=cyan,
}

% *** Citation Packages ***
\usepackage[
	citecounter,
	labelnumber,
	backend=biber,
	bibencoding=utf8,
	sorting=none
]{biblatex}
\addbibresource{bibliography.bib}

% *** Code Blocks Packages ***
\usepackage{listings}
\usepackage{xcolor}
%New colors defined below
\definecolor{codegreen}{rgb}{0,0.6,0}
\definecolor{codegray}{rgb}{0.5,0.5,0.5}
\definecolor{codepurple}{rgb}{0.58,0,0.82}
\definecolor{backcolour}{rgb}{1.0,1.0,1.0}
\lstset{
	frame=tb,
	backgroundcolor=\color{backcolour},
	commentstyle=\color{codegreen},
	keywordstyle=\color{magenta},
	numberstyle=\tiny\color{codegray},
	stringstyle=\color{codepurple},
	basicstyle=\ttfamily\tiny,
	breakatwhitespace=false,
	breaklines=true,
	captionpos=b,
	keepspaces=true,
	numbersep=5pt,
	numbers=left,
	showspaces=false,
	showstringspaces=false,
	showtabs=false,
	tabsize=2
}
% *** Diagram Drawing ***
\usepackage{tikz}
\usetikzlibrary{arrows.meta}
% *** File Organization Packages ***
\usepackage{subfiles}
% *** FONT PACKAGES ***
\usepackage[T1]{fontenc}
% *** MATH PACKAGES ***
\usepackage{amsmath}
\interdisplaylinepenalty=2500
\usepackage{amssymb}
% *** GRAPHICS PACKAGES ***
\usepackage [export]{adjustbox} % Frame around figures (via the frame option in includegraphic)
\usepackage{array}
\usepackage{graphicx}
\usepackage{media9}
\usepackage[font=small,skip=0pt]{caption}
\usepackage{tabularx}
\usepackage{subcaption}

% *** Table Packages ***
\usepackage{multirow}

% *** Unsorted Packages ***
\usepackage[absolute,overlay]{textpos}
	\setlength{\TPHorizModule}{1mm}
	\setlength{\TPVertModule}{1mm}
%\newcolumntype{b}{X}
\newcolumntype{s}{>{\hsize=.5\hsize}X}

% Title Page Info %%%%%%%%%%%%%%%%%%%%%%%%%%%%%%
\title{Emergent Penguin Swarm Behaviors}
\subtitle{
    Final Project for EECE7065 Complex Systems
}
\author{
	\texorpdfstring{
		Presented by: \\
		Wayne Stegner \and Zuguang Liu \and Sid Barve
	}{Wayne Stegner, Zuguang Liu, and Sid Barve}
}

\date{\today}
\logo{\includegraphics[height=1.0cm]{images/uc_logo.png}}

% Start of Document
\begin{document}
	\section{Introduction}
	\begin{frame}
		\centering
		\titlepage
	\end{frame}

	\section{Sid's Update}
	\begin{frame}{Heat Exchange Model}
		Goal: Develop preliminary penguin agents and rules for movement.
		Implement heat exchange interactions among penguins and between penguins
		and the environment.
		\begin{itemize}
			\item Metabolic Heat Generation
				\begin{itemize}
					\item $Q_{meta}=mc\Delta t_{meta}$
				\end{itemize}
			\item Penguin-Penguin Heat Exchange
				\begin{itemize}
					\item $Q_{21}=\frac{\kappa_{1}\kappa_{2}A_{1}A_{2}(t_{2}-t_{1})}{d^{2}}$
				\end{itemize}
			\item Penguin-Environment Heat Exchange (Newton's Law of Cooling)
				\begin{itemize}
					\item $Q_{env}=\kappa A (t_{amb}-t)$
				\end{itemize}
			\item Temperature Change and Movement
				\begin{itemize}
					\item $\Delta t_{1}=\frac{\sum_{i=1}^{N} Q_{i1}+Q_{meta,1}+Q_{env,1}}{m_{1}c_{1}}$
					\item If temp < lower bound, then move toward average location of
						penguins in neighborhood. If temp > upper bound, then move away
						from average location.
				\end{itemize}
		\end{itemize}
	\end{frame}

	\section{Liu's Update}
	\begin{frame}{Temperature Sensing Only Simulation}
		Goal: A simple test simulation on if temperature solely determines the
		emergent swarming behavior
		\begin{itemize}
			\item Heat transfer: penguins radiate heat into the environment with a
				linearly decaying temperature pattern
			\item Moving: if hot, go to cold area; if cold, go to hot area
			\item Penguins are ``blind'': they only sense temperature around them and
				decide on their movement
		\end{itemize}
		Conclusion: ``blind'' penguins will emergently swarm only with a strict
		range of hyper-parameters
	\end{frame}

	\section{Wayne's Update}
	\begin{frame}{Simulation Environment}
		\par Goal: create a framework to facilitate testing different movement
		strategies and model parameters \\
		\begin{itemize}
			\item Framework to manage overall flow of the simulation
				\begin{itemize}
					\item Read parameters from a config file (allows batch processing)
					\item Allow for different agent types modularly
				\end{itemize}
			\item In each epoch:
				\begin{itemize}
					\item Get movement decisions from each agent one at a time
					\item Ignore any moves which cause collisions
					\item Apply thermal model and update agents temperatures
					\item Log metrics and save an image
				\end{itemize}
			\item At the end of each simulation:
				\begin{itemize}
					\item Save a GIF of the simulation
					\item Display metrics such as survival and average temperature
				\end{itemize}
		\end{itemize}
	\end{frame}

\section{Future Work}
\begin{frame}{Future Work}
	Goal: Integrate thermal model and penguin controls in discretized environment.
	\begin{itemize}
		\item Implement thermal model in new discretized environment with cell-cell
			interaction.
			\begin{itemize}
				\item Each cell will have a heat transfer interaction with adjacent
					cells
				\item Penguin cores will generate heat using metabolic heat generation
				\item Different thermal conductivity for penguin interior, penguin
					exterior, and environment
				\item Ambient air or environmental heat sink will remain
			\end{itemize}
		\item Analyze the effects of hyper-parameters and agent controls.
	\end{itemize}
\end{frame}
\begin{frame}{Behavior Evolution}
	\par Goal: Explore evolution for generating rule policies for agents
	\vspace{12pt}
	\par Two possible approaches:
	\begin{itemize}
		\item Use evolution to explore parameters in our current model, keeping our
			rules the same
			\begin{itemize}
				\item Easy with our current infrastructure
				\item Decision models will be all the same
			\end{itemize}
		\item Use evolution to create the rules and parameters
			\begin{itemize}
				\item Use neural nodes with various functions like Sims
					\cite{simsEvolvingVirtualCreatures1994}
				\item Ability to generate interesting and novel decision models
				\item May be difficult to implement --- requires more infrastructure
			\end{itemize}
	\end{itemize}
\end{frame}

	\begin{frame}[allowframebreaks]
		\frametitle{References}
		\printbibliography
	\end{frame}
\end{document}
